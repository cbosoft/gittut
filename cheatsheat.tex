\documentclass{article}
\usepackage[left=1in, right=1in]
{geometry}
\title{Git Cheat Sheet}
\date{ }
\begin{document}
\maketitle
\section*{New repository}

Creating a git repo is the most annoying step to using git, here are the steps I use

\begin{enumerate}
\item Create repo in folder you want to control with git:\newline
  \texttt{git init}
\item Add files to the repo with:\newline
  \textt{git add}
\item Commit changes to the repo with:\newline
  \textt{git commit}\newline
  Your text editor will open up and you will have to write something about what changes you've done, since this is your first commit, something like ``initial commit'' is what I'd put.
\item If you're using git hub, now you have to create the empty repository --- do not add a README just yet! (It will make things a little harder later)
\item Now you have to tell your local git repo where to upload your files:\newline
  \texttt{git remote add origin https://github.com/YOURGITUSERNAME/YOURGITREPO}
\item Then you push the local changes up to the git repo:\newline
  \textt{git push --set-upstream origin master}\newline
  You will be asked to login and can then start using your repo as normal!
\end{enumerate}
  
\section*{Updating the Repository}
Whenever you make changes to the repo, do the following to upload those changes to github:
\begin{enumerate}
\item Add any new files:\newline
  \texttt{git add <FILE>}
\item Commit changes made to files:\newline
  \texttt{git commit <FILE>}
\item Then push to server:\newline
  \texttt{git push}
\end{enumerate}

\end{document}
